\documentclass[a4paper,14pt,draft]{memoir}
\usepackage{geometry}
\usepackage[french]{babel}
\usepackage{xltxtra}
\usepackage{graphicx,array}
\usepackage{hyperref}
\usepackage{biblatex}
\usepackage[refpage]{nomencl}
\addbibresource{biblio.bib}
\newcolumntype{C}[1]{>{\centering\let\newline\\\arraybackslash\hspace{0pt}}m{#1}}
\providecommand{\tightlist}{%
  \setlength{\itemsep}{0pt}\setlength{\parskip}{0pt}}

\makenomenclature

\title{Construction et analyse de tests adaptatifs dans un cadre de filtrage collaboratif}
\author{Jill-Jênn Vie}

\begin{document}
\maketitle

\chapter*{Remerciements}
\input{merci}

\printnomenclature

\setsecnumdepth{chapter}
\setcounter{tocdepth}{0}

\chapter*{Résumé}
\input{summary}

\setsecnumdepth{subsection}
\setcounter{tocdepth}{2}

\clearpage
\tableofcontents

\chapter{Introduction}
\input{intro}

\chapter{État de l'art}
\input{adaptive-full}

\chapter{Framework de comparaison de modèles de tests adaptatifs}
\input{intro-framework}
\input{comparison}

\chapter{GenMA : un modèle hybride adaptatif de diagnostic de connaissances}
\input{intro-genma}
\input{factorization}
\documentclass{standalonex}
\usepackage{tikz}

\begin{document}
\begin{tikzpicture}
\draw (-0.5,-0.5) rectangle ++(5,6);
\node at (0,0) {0.20};
\node at (1,0) {0.25};
\node at (2,0) {0.54};
\node at (3,0) {0.60};
\node at (4,0) {0.12};
\node at (0,1) {0.33};
\node at (1,1) {0.33};
\node at (2,1) {0.05};
\node at (3,1) {0.61};
\node at (4,1) {0.06};
\node at (0,2) {0.53};
\node at (1,2) {0.84};
\node at (2,2) {0.80};
\node at (3,2) {0.34};
\node at (4,2) {0.01};
\node at (0,3) {0.06};
\node at (1,3) {0.99};
\node at (2,3) {0.19};
\node at (3,3) {0.98};
\node at (4,3) {0.26};
\node at (0,4) {0.50};
\node at (1,4) {0.11};
\node at (2,4) {0.16};
\node at (3,4) {0.39};
\node at (4,4) {0.91};
\node at (0,5) {0.15};
\node at (1,5) {0.52};
\node at (2,5) {0.37};
\node at (3,5) {0.22};
\node at (4,5) {0.12};
\node at (2,-1) {MIRT};
\node at (5,2.5) {$\cup$};
\begin{scope}[xshift=6cm]
\draw (-0.5,-0.5) rectangle ++(5,6);
\node at (0,0) {1};
\node at (1,0) {0};
\node at (2,0) {1};
\node at (3,0) {1};
\node at (4,0) {1};
\node at (0,1) {1};
\node at (1,1) {1};
\node at (2,1) {0};
\node at (3,1) {0};
\node at (4,1) {0};
\node at (0,2) {0};
\node at (1,2) {0};
\node at (2,2) {0};
\node at (3,2) {1};
\node at (4,2) {1};
\node at (0,3) {1};
\node at (1,3) {0};
\node at (2,3) {1};
\node at (3,3) {0};
\node at (4,3) {1};
\node at (0,4) {1};
\node at (1,4) {0};
\node at (2,4) {1};
\node at (3,4) {1};
\node at (4,4) {0};
\node at (0,5) {1};
\node at (1,5) {1};
\node at (2,5) {0};
\node at (3,5) {1};
\node at (4,5) {1};
\node at (2,-1) {Q-matrice};
\end{scope}
\node at (11,2.5) {$\Rightarrow$};
\begin{scope}[xshift=12cm]
\draw (-0.5,-0.5) rectangle ++(5,6);
\node at (0,0) {0.63};
\node at (2,0) {0.71};
\node at (3,0) {1.00};
\node at (4,0) {0.00};
\node at (0,1) {0.25};
\node at (1,1) {0.54};
\node at (3,2) {0.41};
\node at (4,2) {0.10};
\node at (0,3) {0.03};
\node at (2,3) {0.05};
\node at (4,3) {0.36};
\node at (0,4) {0.05};
\node at (2,4) {0.27};
\node at (3,4) {0.07};
\node at (0,5) {0.47};
\node at (1,5) {0.18};
\node at (3,5) {0.15};
\node at (4,5) {0.63};
\node at (2,-1) {GenMA};
\end{scope}
\end{tikzpicture}
\end{document}


\chapter{InitialD : une heuristique pour le démarrage à froid d'un test}
\input{intro-dpp}
\input{recommenders}
\input{dpp}

\chapter{Conclusion et perspectives}
\input{perspectives}

\appendix

\chapter{Méthodologie : mise en œuvre dans un MOOC de Coursera}
\input{intro-mooc}
\input{mooc}

\chapter{Lien entre théorie de la réponse à l'item et apprentissage profond}
\input{rbm}

\printbibliography

\end{document}
