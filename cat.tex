\documentclass[12pt,a4paper]{book}
\usepackage{ifpdf,ifxetex,ifluatex}
\usepackage[T1]{fontenc}
\usepackage[a-1b]{pdfx}
\usepackage{xcolor}
\usepackage{hyperref}
\ifxetex
\usepackage{xltxtra}
\fi
\ifluatex
\usepackage{mathtools}
\usepackage{unicode-math}
\usepackage{libertine}
\setmathfont{TeX Gyre Pagella Math}
\setmathfont[BoldFont=TeX Gyre Pagella Bold Italic,range=\mathit/{latin,Latin,greek,Greek,num}]{TeX Gyre Pagella Italic}
\setmathfont[range=\setminus]{Asana Math}
%\usepackage[libertine={Ligatures=TeX,RawFeature=+onum},
%            biolinum={Ligatures=TeX,RawFeature=+onum}]{libertineotf}
\setmonofont[Scale=MatchLowercase]{DejaVu Sans Mono}
\usepackage{microtype}
\fi
\usepackage[english,french]{babel}
\DecimalMathComma
\usepackage{fancyhdr}
\usepackage[pass]{geometry}
\usepackage{algorithm}
\makeatletter
\renewcommand{\ALG@name}{Algorithme}
\makeatother
\usepackage{caption,algpseudocode}
\DeclareCaptionLabelFormat{bsc}{\textbf{\textsc{#1}\ #2}}
\captionsetup[figure]{labelformat=bsc}
\captionsetup[table]{labelformat=bsc}
\usepackage{graphicx,array,booktabs,multirow}
\usepackage[french]{varioref}
\usepackage{multicol}
\usepackage[french=guillemets]{csquotes}

\usepackage[a-1b]{pdfx}
% \hypersetup{colorlinks=true,linkcolor=black,citecolor=blue}

\usepackage[style=authoryear,natbib=true,backend=biber,backref=true,maxcitenames=3,maxbibnames=99,babel=other]{biblatex}
\usepackage[french=guillemets]{csquotes}
\usepackage{capt-of}
\DeclareLanguageMapping{french}{french-apa}
\DefineBibliographyExtras{french}{\restorecommand\mkbibnamefamily}
\usepackage[refpage]{nomencl}
\usepackage[acronym]{glossaries}
\addbibresource{biblio.bib}
\setcounter{tocdepth}{2}
\newfontfamily{\J}{IPAexMincho}
\def\Tomoko{Tomoko Kōzu ({\J 神津智子})}
\def\Tito{Lê Thành D\~{u}ng Nguy\~{ê}n}
\def\Gambaru{{\J 頑張る}}

\definecolor{aquali}{cmyk}{1,0.0474,0,0.255}
\hypersetup{colorlinks=true,urlcolor=aquali,linkcolor=black,citecolor=blue}

% \usepackage{titlesec}
% \titleformat{\chapter}[display]
% {\hipsterfont\huge\bfseries}{\chaptertitlename\ \thechapter}{20pt}{\Huge}
% \titleformat{\section}
% {\hipsterfont\Large\bfseries}{\thesection}{1em}{}
% \titleformat{\subsection}
% {\hipsterfont\large\bfseries}{\thesubsection}{1em}{}

\renewcommand{\chaptermark}[1]{\markboth{#1}{}}
\renewcommand{\sectionmark}[1]{\markright{#1}}
\pagestyle{fancy}
\fancyhf{}
\fancyhead[LE,RO]{\thepage}
\fancyhead[LO]{\nouppercase{\rightmark}}
\fancyhead[RE]{\nouppercase{\leftmark}}
\renewcommand{\headrulewidth}{0pt}
\setlength{\headheight}{15pt}

\newtheorem{algo}[equation]{Algorithme}
\algtext*{EndProcedure}
\algtext*{EndWhile}
\algtext*{EndFor}
\algtext*{EndIf}
\algrenewcommand\algorithmicprocedure{\textbf{procédure}}
\algrenewcommand\algorithmicif{\textbf{si}}
\algrenewcommand\algorithmicthen{\textbf{alors}}
\algrenewcommand\algorithmicelse{\textbf{sinon}}
\algrenewcommand\algorithmicfor{\textbf{pour}}
\algrenewcommand\algorithmicwhile{\textbf{tant que}}
\algrenewcommand\algorithmicdo{\textbf{faire}}
\algrenewcommand\algorithmicreturn{\textbf{renvoyer}}

\newcolumntype{C}[1]{>{\centering\let\newline\\\arraybackslash\hspace{0pt}}m{#1}}
\providecommand{\tightlist}{%
  \setlength{\itemsep}{0pt}\setlength{\parskip}{0pt}}

\makeglossaries
\def\pagedeclaration#1{, \hyperlink{page.#1}{page\nobreakspace#1}}
\newcommand\englishpub[1]{{\begin{otherlanguage}{english}#1\end{otherlanguage}}}
\makenomenclature

\title{Modèles de tests adaptatifs pour le diagnostic de connaissances dans un cadre d'apprentissage à grande échelle}
\author{Jill-Jênn Vie}

%%% This LaTeX template was created by Adrien Thurotte
%% It is approved by Paris-Saclay University (december 2015)
% And, of course, needs improvements : feel free to work on it and share it back ...
%% Under the licence CC0 - Others may freely build upon, enhance and reuse the works for any purposes without restriction under copyright or database law. https://creativecommons.org/about/cc0
%\usepackage[T1]{fontenc}
%\usepackage[latin1]{inputenc}
%\usepackage[french]{babel}
%\usepackage{graphicx} % For commande \includegraphics
%\usepackage[usenames,dvipsnames,svgnames,table]{xcolor} %% To color in violet the text
%\frenchspacing
\usepackage[absolute]{textpos} % to place elements in the page
%\usepackage{multicol} % To write summary in two columns-mode
%\usepackage{calc} % To calculate textwidth
%\RequirePackage{geometry}% That nicely create a one-page template
\usepackage{tikz} %% Make the square.

\label{The thesis}
\newcommand{\PhDTitleFR}{Modèles de tests adaptatifs pour le diagnostic de connaissances dans un cadre d’apprentissage à grande échelle} %% Titre en Français
\newcommand{\PhDkeywordsFR}{Circummurana, Pertulit, Bella (3-6 mots-clefs)} %% 3 à 6 mots clefs
\newcommand{\PhDsumFR}{Eius populus ab incunabulis primis ad usque pueritiae tempus extremum, quod annis circumcluditur fere trecentis, circummurana pertulit bella, deinde aetatem ingressus adultam post multiplices bellorum aerumnas Alpes transcendit et fretum, in iuvenem erectus et virum ex omni plaga quam orbis ambit inmensus, reportavit laureas et triumphos, iamque vergens in senium et nomine solo aliquotiens vincens ad tranquilliora vitae discessit. 

Hoc inmaturo interitu ipse quoque sui pertaesus excessit e vita aetatis nono anno atque vicensimo cum quadriennio imperasset. natus apud Tuscos in Massa Veternensi, patre Constantio Constantini fratre imperatoris, matreque Galla.   Thalassius vero ea tempestate praefectus praetorio praesens ipse quoque adrogantis ingenii, considerans incitationem eius ad multorum augeri discrimina, non maturitate vel consiliis mitigabat, ut aliquotiens celsae potestates iras principum molliverunt, sed adversando iurgandoque cum parum congrueret, eum ad rabiem potius evibrabat, Augustum actus eius exaggerando creberrime docens, idque, incertum qua mente, ne lateret adfectans. quibus mox Caesar acrius efferatus, velut contumaciae quoddam vexillum altius erigens, sine respectu salutis alienae vel suae ad vertenda opposita instar rapidi fluminis irrevocabili impetu ferebatur.

Hae duae provinciae bello quondam piratico catervis mixtae praedonum.} %% Résumé


\newcommand{\PhDTitleEN}{Eius populus ab incunabulis primis ad usque pueritiae tempus extremum, quod annis circumcluditur fere trecentis, deinde aetatem ingressus} %% Title
\newcommand{\PhDkeywordsEN}{circummurana, pertulit, bella (3-6 keywords)} %% 3-6 Keywords
\newcommand{\PhDsumEN}{Eius populus ab incunabulis primis ad usque pueritiae tempus extremum, quod annis circumcluditur fere trecentis, circummurana pertulit bella, deinde aetatem ingressus adultam post multiplices bellorum aerumnas Alpes transcendit et fretum, in iuvenem erectus et virum ex omni plaga quam orbis ambit inmensus, reportavit laureas et triumphos, iamque vergens in senium et nomine solo aliquotiens vincens ad tranquilliora vitae discessit. 

Hoc inmaturo interitu ipse quoque sui pertaesus excessit e vita aetatis nono anno atque vicensimo cum quadriennio imperasset. natus apud Tuscos in Massa Veternensi, patre Constantio Constantini fratre imperatoris, matreque Galla.   Thalassius vero ea tempestate praefectus praetorio praesens ipse quoque adrogantis ingenii, considerans incitationem eius ad multorum augeri discrimina, non maturitate vel consiliis mitigabat, ut aliquotiens celsae potestates iras principum molliverunt, sed adversando iurgandoque cum parum congrueret, eum ad rabiem potius evibrabat, Augustum actus eius exaggerando creberrime docens, idque, incertum qua mente, ne lateret adfectans. quibus mox Caesar acrius efferatus, velut contumaciae quoddam vexillum altius erigens, sine respectu salutis alienae vel suae ad vertenda opposita instar rapidi fluminis irrevocabili impetu ferebatur.

Hae duae provinciae bello quondam piratico catervis mixtae praedonum.} %% Summary

%%%%%%%%%%%%%%%%%%%%%%%%%%%%%%%%%%%%%%%%%%%%%%%%%%%%%%%%%%%%%%%%%%%%%%%%%%%%%%%%%%%%%%%%%%%%%%%%% 
 
 \label{Personnal}
\newcommand{\PhDname}{M. Jill-Jênn Vie} % Civility, first name and name 
\newcommand{\NNT}{2016SACLC090} %% Your NNT numer (the Library will attribute one...)
\newcommand{\ecodocnum}{580} % Accrediation number
\newcommand{\ecodoctitle}{ED Sciences et technologies de l'information et de la communication} % Full name of the doctorale school
\newcommand{\PhDspeciality}{Informatique} % Speciality 
\newcommand{\PhDworkingplace}{CentraleSupélec} %PhD working place (must be one of these : Université Paris-Sud, Université de Versailles Saint Quentin, Université d’Evry, AgroParisTech, CentraleSupelec, Ecole Normale Supérieure de Cachan, Ecole Polytechnique, ENSTA ParisTech,ENSAE ParisTech, HEC,Institut d’Optique Graduate School,Telecom ParisTech or Telecom SudParis)
\newcommand{\defenseplace}{l'École normale supérieure de Paris-Saclay} %Place of defense
\newcommand{\defensedate}{5 décembre 2016} % Date of defense
\newcommand{\logoED}{\includegraphics[scale=1.2]{logo/STIC.jpg}} % Logo of doctoral school. Check the name of the correct file in the /logo folder.
\newcommand{\logoEt}{\includegraphics[scale=0.7]{logo/CentraleSupelec.png}} % Must be the logo corresponding to %PhD working place. Check the name of the correct file in the /logo folder.
\newcommand{\vpos}{1.4} %% You can modify vertical position (leave cm as unit) in order to align horizontally both images

%%%%%%In a case of "cotutelle" Delete " % " of the three following lines. Then, in style-pagedegarde.tex, delete " % " of line 3 to 5.
%\newcommand{\logoCotut}{\includegraphics[scale=1]{logo/UPS.png}} % Add here the logo of the partner. 
%\newcommand{\vCotutpos}{3.3} %% You can modify vertical position (leave cm as unit) in order to align horizontally both images
%\newcommand{\hCotutpos}{12.5} %% You can modify horizontal position (leave cm as unit) in order to align horizontally both images


%%%%%%%%%%%%%%%%%%%%%%%%%%%%%%%%%%%%%%%JURY%%%%%%%%%%%%%%%%%%%%%%%%%%%%%%%%%%%%%%%%%%%%%%%%%%%%%%%%%%%%%
% Lors du premier dépôt de la thèse le nom du président n’est pas connu, le choix du président se fait par les membres du Jury juste avant la soutenance. La précision est apportée sur la couverture lors du second dépôt.
% Choice of the jury's president is made during the defense. Thus, it must be specified only for the second file deposition in ADUM.
% All the jury members listed here must have been present during the defense.
\label{Jury}
%%% Jury member n1 (Président) %%%
\newcommand{\jurynameA}{Amel Bouzeghoub}
\newcommand{\jurygenderA}{Mme} % M. or Mme. / Mrs.
\newcommand{\juryadressA}{Télécom SudParis}
\newcommand{\jurygradeA}{Professeur}
\newcommand{\juryroleA}{Présidente du jury} % 
%%% Jury member n2 (Rapporteur) %%%
\newcommand{\jurynameB}{Nathalie Guin}
\newcommand{\jurygenderB}{Mme} % M. or Mme. / Mrs.
\newcommand{\juryadressB}{LIRIS}
\newcommand{\jurygradeB}{Maître de conférences HDR}
\newcommand{\juryroleB}{Rapporteur}
%%% Jury member n3 (Rapporteur) %%%
\newcommand{\jurynameC}{Sébastien George}
\newcommand{\jurygenderC}{M.} % M. or Mme. / Mrs.
\newcommand{\juryadressC}{Université du Maine}
\newcommand{\jurygradeC}{Professeur des universités}
\newcommand{\juryroleC}{Rapporteur}
%%% Jury member n4 (Examinateur) %%%
\newcommand{\jurynameD}{Vanda Luengo}
\newcommand{\jurygenderD}{Mme} % M. or Mme. / Mrs.
\newcommand{\juryadressD}{UPMC}
\newcommand{\jurygradeD}{Professeur des universités}
\newcommand{\juryroleD}{Examinatrice}
%%% Jury member n5 (Examinateur) %%%
\newcommand{\jurynameE}{Monique Grandbastien}
\newcommand{\jurygenderE}{Mme} % M. or Mme. / Mrs.
\newcommand{\juryadressE}{LORIA}
\newcommand{\jurygradeE}{Professeur émérite}
\newcommand{\juryroleE}{Examinatrice}
%%% Jury member n6 (Examinateur) %%%
\newcommand{\jurynameF}{Yolaine Bourda}
\newcommand{\jurygenderF}{Mme} % M. or Mme. / Mrs.
\newcommand{\juryadressF}{CentraleSupélec}
\newcommand{\jurygradeF}{Professeur}
\newcommand{\juryroleF}{Directrice de thèse}
%%% Jury member n7 (Examinateur) %%%
\newcommand{\jurynameG}{Éric Bruillard}
\newcommand{\jurygenderG}{M.} % M. or Mme. / Mrs.
\newcommand{\juryadressG}{ENS Paris-Saclay}
\newcommand{\jurygradeG}{Professeur des universités}
\newcommand{\juryroleG}{Codirecteur de thèse}
%%% Jury member n8 (Examinateur) %%%
\newcommand{\jurynameH}{Fabrice Popineau}
\newcommand{\jurygenderH}{M.} % M. or Mme. / Mrs.
\newcommand{\juryadressH}{CentraleSupélec}
\newcommand{\jurygradeH}{Professeur}
\newcommand{\juryroleH}{Coencadrant de thèse}
%% Il est possible d'ajouter des membres supplémentaires selon le même modèle.
%% More jury members can be added according to the same model
%%%%%%%%%%%%%%%%%%%%%%%%%%%%%%%%%%%%%%%COMPILATION%%%%%%%%%%%%%%%%%%%%%%%%%%%%%%%%%%%%%%%%%%%%%%%%%%%%%%%%%%%%%
\label{Document}
%% 
%% Les fichiers sont à compiler un par un. Il faut retirer la marque de commentaire " % " en début de ligne, compiler le document, puis mettre la ligne à nouveau en commentaire en vue de compiler le document suivant.
%% To compile desired document, delete the " % " at the begin of the line, compile the file, and then comment again in order to be able to compile the next one.
%%


\begin{document}
\newgeometry{textheight=150ex,textwidth=40em,top=30pt,headheight=30pt,headsep=30pt,inner=80pt}
%%%%%%%%%%%%%%%%%%%%%%%%%%%%%COCUTELLE%%%%%%%%%%%%%%%%%%%%%%%%%%%%%%%%%%%%%%%%%%%%%%%
\label{cotutelle}
%\begin{textblock}{1}(\hCotutpos,\vCotutpos)
%  \logoCotut %% Logo en cas de cotutelle
%\end{textblock}
%%%%%%%%%%%%%%%%%%%%%%%%%%%%%COCUTELLE%%%%%%%%%%%%%%%%%%%%%%%%%%%%%%%%%%%%%%%%%%%%%%%




%% Positionner le cadre dans la page.
	%% Modifier yshift modifie la position des bords haut et bas du cadre. Modifier xshift modifie la position des bords gauche et droit du cadre. Il faut toujours les modifier deux par deux (ceux qui ont la même valeur ensemble).
\begin{tikzpicture}[remember picture,overlay,color=blue!20!red!45!black!75!]
	\draw[very thick]
		([yshift=-160pt,xshift=45pt]current page.north west)--     
		([yshift=-160pt,xshift=-25pt]current page.north east)--    
		([yshift=35pt,xshift=-25pt]current page.south east)--      
		([yshift=35pt,xshift=45pt]current page.south west)--cycle; 
\end{tikzpicture}


%% Position du NNT
\begin{textblock}{13}(1.15,3.3)
  NNT : \NNT
\end{textblock}


%% Logos en haut de la page
\begin{textblock}{1}(1.15,1)
\includegraphics[height=2.4cm]{logo/UPSac.png} %% Logo de Paris Saclay
\label{Logo Paris Saclay}
\end{textblock}

\begin{textblock}{1}(12,\vpos)
\logoEt %% Logo de votre établissement
\label{Logo Etablissement}
\end{textblock}

\vspace{6cm}
%% Texte
\color{blue!20!red!45!black} %% Couleur violette du premier paragraphe
  \begin{center}    
    \LARGE\textsc{Thèse de doctorat\\ de l'Université Paris-Saclay} \\
    \LARGE{\textsc{préparée à \PhDworkingplace}} \\ \bigskip
  \color{black} %% Couleur noir du reste du texte
	\vfill
    \Large{École doctorale \no \ecodocnum}\\ %% Numéro ED
     \Large{\ecodoctitle}  \\

     \Large{Spécialité de doctorat: \PhDspeciality} %% Spécialité
    \vfill  
   \Large{par}
   \vfill
   \LARGE{\textbf{\textsc{\PhDname}}} %% Nom du docteur
    \vfill
    \Large{\PhDTitleFR} %% Titre de la thèse
    \vfill
    \bigskip
\end{center}
\color{black}
%% Jury
\begin{flushleft}
Thèse présentée et soutenue à \defenseplace, le \defensedate. \\
\bigskip
Composition du Jury :
\end{flushleft}
%% Members of the jury
%% If needed, one can add jurymemberG or remove one jury member.

\begin{center}
\begin{tabular}{llll}

    \jurygenderA & \textsc{\jurynameA}  & \jurygradeA & (\juryroleA) \\
    \null & \null & \juryadressA &\\   
   
    \jurygenderB & \textsc{\jurynameB}  & \jurygradeB & (\juryroleB) \\
    \null & \null & \juryadressB &\\ 
    
    \jurygenderC & \textsc{\jurynameC}  & \jurygradeC & (\juryroleC) \\
    \null & \null & \juryadressC &\\ 
    
    \jurygenderD & \textsc{\jurynameD}  & \jurygradeD & (\juryroleD) \\
    \null & \null & \juryadressD &\\ 
    
    \jurygenderE & \textsc{\jurynameE}  & \jurygradeE & (\juryroleE) \\
    \null & \null & \juryadressE &\\ 
    
    \jurygenderF & \textsc{\jurynameF}  & \jurygradeF & (\juryroleF) \\
    \null & \null & \juryadressF &\\ 
   
    \jurygenderG & \textsc{\jurynameG}  & \jurygradeG & (\juryroleG) \\
    \null & \null & \juryadressG &\\ 
   
    \jurygenderH & \textsc{\jurynameH}  & \jurygradeH & (\juryroleH) \\
    \null & \null & \juryadressH &\\ 
   
  \end{tabular}    
\end{center}

\restoregeometry

\chapter*{Remerciements}
\input{merci}

\printglossaries
\printnomenclature

\clearpage
\tableofcontents

\chapter{Introduction}
\input{intro}

\chapter{État de l'art}
\input{adaptive-full}

\chapter{Système de comparaison de modèles de tests adaptatifs}
\input{intro-framework}
\input{comparison}

\chapter{GenMA : un modèle hybride de diagnostic de connaissances}
\input{intro-genma}
\input{factorization}
\documentclass{standalonex}
\usepackage{tikz}

\begin{document}
\begin{tikzpicture}
\draw (-0.5,-0.5) rectangle ++(5,6);
\node at (0,0) {0.20};
\node at (1,0) {0.25};
\node at (2,0) {0.54};
\node at (3,0) {0.60};
\node at (4,0) {0.12};
\node at (0,1) {0.33};
\node at (1,1) {0.33};
\node at (2,1) {0.05};
\node at (3,1) {0.61};
\node at (4,1) {0.06};
\node at (0,2) {0.53};
\node at (1,2) {0.84};
\node at (2,2) {0.80};
\node at (3,2) {0.34};
\node at (4,2) {0.01};
\node at (0,3) {0.06};
\node at (1,3) {0.99};
\node at (2,3) {0.19};
\node at (3,3) {0.98};
\node at (4,3) {0.26};
\node at (0,4) {0.50};
\node at (1,4) {0.11};
\node at (2,4) {0.16};
\node at (3,4) {0.39};
\node at (4,4) {0.91};
\node at (0,5) {0.15};
\node at (1,5) {0.52};
\node at (2,5) {0.37};
\node at (3,5) {0.22};
\node at (4,5) {0.12};
\node at (2,-1) {MIRT};
\node at (5,2.5) {$\cup$};
\begin{scope}[xshift=6cm]
\draw (-0.5,-0.5) rectangle ++(5,6);
\node at (0,0) {1};
\node at (1,0) {0};
\node at (2,0) {1};
\node at (3,0) {1};
\node at (4,0) {1};
\node at (0,1) {1};
\node at (1,1) {1};
\node at (2,1) {0};
\node at (3,1) {0};
\node at (4,1) {0};
\node at (0,2) {0};
\node at (1,2) {0};
\node at (2,2) {0};
\node at (3,2) {1};
\node at (4,2) {1};
\node at (0,3) {1};
\node at (1,3) {0};
\node at (2,3) {1};
\node at (3,3) {0};
\node at (4,3) {1};
\node at (0,4) {1};
\node at (1,4) {0};
\node at (2,4) {1};
\node at (3,4) {1};
\node at (4,4) {0};
\node at (0,5) {1};
\node at (1,5) {1};
\node at (2,5) {0};
\node at (3,5) {1};
\node at (4,5) {1};
\node at (2,-1) {Q-matrice};
\end{scope}
\node at (11,2.5) {$\Rightarrow$};
\begin{scope}[xshift=12cm]
\draw (-0.5,-0.5) rectangle ++(5,6);
\node at (0,0) {0.63};
\node at (2,0) {0.71};
\node at (3,0) {1.00};
\node at (4,0) {0.00};
\node at (0,1) {0.25};
\node at (1,1) {0.54};
\node at (3,2) {0.41};
\node at (4,2) {0.10};
\node at (0,3) {0.03};
\node at (2,3) {0.05};
\node at (4,3) {0.36};
\node at (0,4) {0.05};
\node at (2,4) {0.27};
\node at (3,4) {0.07};
\node at (0,5) {0.47};
\node at (1,5) {0.18};
\node at (3,5) {0.15};
\node at (4,5) {0.63};
\node at (2,-1) {GenMA};
\end{scope}
\end{tikzpicture}
\end{document}


\chapter{InitialD : une heuristique pour le démarrage à froid}
\input{intro-dpp}
\input{dpp}

\chapter{Conclusion et perspectives}
\input{perspectives}

%\chapter*{Résumé}
%\input{summary}

\clearpage
\listoffigures
\listoftables

\appendix

\chapter{Implémentation des modèles}
\input{code}

%\nocite{*}
\emergencystretch=1em
\printbibliography

\null\newpage
\thispagestyle{empty}
\begin{textblock}{10}(2,.8)
\logoED
\end{textblock}
%\vfill


\begin{textblock}{13}(2,3)
\paragraph{Titre :} \textbf{\PhDTitleFR} 
\paragraph{Mots-cl\'es :} \PhDkeywordsFR  \bigskip

\paragraph{R\'esum\'e :} \PhDsumFR 
\end{textblock}


\begin{textblock}{13}(13,14.5)
\includegraphics[height=2cm]{logo/logoEgrey.png}
\end{textblock}
\parindent=0pt 



\begin{textblock}{10}(2.2,14.6)\color{blue!20!red!45!black}{\footnotesize{\textbf{Universit\'e Paris-Saclay}\\Espace Technologique / Immeuble Discovery \\
Route de l'Orme aux Merisiers RD 128 / 91190 Saint-Aubin, France}}

\end{textblock}


\null\newpage
\thispagestyle{empty}

  \begin{textblock}{200}(400,200)
    \centering
    \vspace{20mm}
      { \bfseries \Large Hello World }
    \vspace{20mm}
  \end{textblock}



\begin{textblock}{10}(2,.8)
\logoED
\end{textblock}
%\vfill


\begin{textblock}{13}(2,3)
\paragraph{Title.} \textbf{\PhDTitleEN}
\paragraph{Keywords.}\PhDkeywordsEN  \bigskip

\paragraph{Abstract.} \PhDsumEN
\end{textblock}


\begin{textblock}{13}(13,14.5)
\includegraphics[height=2cm]{logo/logoEgrey.png}
\end{textblock}
\parindent=0pt 

\begin{textblock}{10}(2.2,14.6)\color{blue!20!red!45!black}{\footnotesize{\textbf{Universit\'e Paris-Saclay}\\Espace Technologique / Immeuble Discovery \\
Route de l'Orme aux Merisiers RD 128 / 91190 Saint-Aubin, France}}

\end{textblock}





\end{document}

%%% Local Variables:
%%% mode: latex
%%% TeX-master: t
%%% End:
