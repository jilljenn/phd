\documentclass[12pt,a4paper]{book}
\usepackage{ifpdf,ifxetex,ifluatex}
\usepackage[T1]{fontenc}
\usepackage[a-1b]{pdfx}
\usepackage{xcolor}
\usepackage{hyperref}
\ifxetex
\usepackage{xltxtra}
\fi
\ifluatex
\usepackage{mathtools}
\usepackage{unicode-math}
\usepackage{libertine}
\setmathfont{TeX Gyre Pagella Math}
\setmathfont[BoldFont=TeX Gyre Pagella Bold Italic,range=\mathit/{latin,Latin,greek,Greek,num}]{TeX Gyre Pagella Italic}
\setmathfont[range=\setminus]{Asana Math}
%\usepackage[libertine={Ligatures=TeX,RawFeature=+onum},
%            biolinum={Ligatures=TeX,RawFeature=+onum}]{libertineotf}
\setmonofont[Scale=MatchLowercase]{DejaVu Sans Mono}
\usepackage{microtype}
\fi
\usepackage[english,french]{babel}
\DecimalMathComma
\usepackage{fancyhdr}
\usepackage[pass]{geometry}
\usepackage{algorithm}
\makeatletter
\renewcommand{\ALG@name}{Algorithme}
\makeatother
\usepackage{caption,algpseudocode}
\DeclareCaptionLabelFormat{bsc}{\textbf{\textsc{#1}\ #2}}
\captionsetup[figure]{labelformat=bsc}
\captionsetup[table]{labelformat=bsc}
\usepackage{graphicx,array,booktabs,multirow}
\usepackage[french]{varioref}
\usepackage{multicol}
\usepackage[french=guillemets]{csquotes}

\usepackage[a-1b]{pdfx}
% \hypersetup{colorlinks=true,linkcolor=black,citecolor=blue}

\usepackage[style=authoryear,natbib=true,backend=biber,backref=true,maxcitenames=3,maxbibnames=99,babel=other]{biblatex}
\usepackage[french=guillemets]{csquotes}
\usepackage{capt-of}
\DeclareLanguageMapping{french}{french-apa}
\DefineBibliographyExtras{french}{\restorecommand\mkbibnamefamily}
\usepackage[refpage]{nomencl}
\usepackage[acronym]{glossaries}
\addbibresource{biblio.bib}
\setcounter{tocdepth}{2}
\newfontfamily{\J}{IPAexMincho}
\def\Tomoko{Tomoko Kōzu ({\J 神津智子})}
\def\Tito{Lê Thành D\~{u}ng Nguy\~{ê}n}
\def\Gambaru{{\J 頑張る}}

\definecolor{aquali}{cmyk}{1,0.0474,0,0.255}
\hypersetup{colorlinks=true,urlcolor=aquali,linkcolor=black,citecolor=blue}

% \usepackage{titlesec}
% \titleformat{\chapter}[display]
% {\hipsterfont\huge\bfseries}{\chaptertitlename\ \thechapter}{20pt}{\Huge}
% \titleformat{\section}
% {\hipsterfont\Large\bfseries}{\thesection}{1em}{}
% \titleformat{\subsection}
% {\hipsterfont\large\bfseries}{\thesubsection}{1em}{}

\renewcommand{\chaptermark}[1]{\markboth{#1}{}}
\renewcommand{\sectionmark}[1]{\markright{#1}}
\pagestyle{fancy}
\fancyhf{}
\fancyhead[LE,RO]{\thepage}
\fancyhead[LO]{\nouppercase{\rightmark}}
\fancyhead[RE]{\nouppercase{\leftmark}}
\renewcommand{\headrulewidth}{0pt}
\setlength{\headheight}{15pt}

\newtheorem{algo}[equation]{Algorithme}
\algtext*{EndProcedure}
\algtext*{EndWhile}
\algtext*{EndFor}
\algtext*{EndIf}
\algrenewcommand\algorithmicprocedure{\textbf{procédure}}
\algrenewcommand\algorithmicif{\textbf{si}}
\algrenewcommand\algorithmicthen{\textbf{alors}}
\algrenewcommand\algorithmicelse{\textbf{sinon}}
\algrenewcommand\algorithmicfor{\textbf{pour}}
\algrenewcommand\algorithmicwhile{\textbf{tant que}}
\algrenewcommand\algorithmicdo{\textbf{faire}}
\algrenewcommand\algorithmicreturn{\textbf{renvoyer}}

\newcolumntype{C}[1]{>{\centering\let\newline\\\arraybackslash\hspace{0pt}}m{#1}}
\providecommand{\tightlist}{%
  \setlength{\itemsep}{0pt}\setlength{\parskip}{0pt}}

\makeglossaries
\def\pagedeclaration#1{, \hyperlink{page.#1}{page\nobreakspace#1}}
\newcommand\englishpub[1]{{\begin{otherlanguage}{english}#1\end{otherlanguage}}}
\makenomenclature

\title{Modèles de tests adaptatifs pour le diagnostic de connaissances dans un cadre d'apprentissage à grande échelle}
\author{Jill-Jênn Vie}

\input{ups}

\begin{document}
\newgeometry{textheight=150ex,textwidth=40em,top=30pt,headheight=30pt,headsep=30pt,inner=80pt}
\input{style-pagedegarde}
\restoregeometry

\chapter*{Remerciements}
\input{merci}

\printglossaries
\printnomenclature

\clearpage
\tableofcontents

\chapter{Introduction}
\input{intro}

\chapter{État de l'art}
\input{adaptive-full}

\chapter{Système de comparaison de modèles de tests adaptatifs}
\input{intro-framework}
\input{comparison}

\chapter{GenMA : un modèle hybride de diagnostic de connaissances}
\input{intro-genma}
\input{factorization}
\documentclass{standalonex}
\usepackage{tikz}

\begin{document}
\begin{tikzpicture}
\draw (-0.5,-0.5) rectangle ++(5,6);
\node at (0,0) {0.20};
\node at (1,0) {0.25};
\node at (2,0) {0.54};
\node at (3,0) {0.60};
\node at (4,0) {0.12};
\node at (0,1) {0.33};
\node at (1,1) {0.33};
\node at (2,1) {0.05};
\node at (3,1) {0.61};
\node at (4,1) {0.06};
\node at (0,2) {0.53};
\node at (1,2) {0.84};
\node at (2,2) {0.80};
\node at (3,2) {0.34};
\node at (4,2) {0.01};
\node at (0,3) {0.06};
\node at (1,3) {0.99};
\node at (2,3) {0.19};
\node at (3,3) {0.98};
\node at (4,3) {0.26};
\node at (0,4) {0.50};
\node at (1,4) {0.11};
\node at (2,4) {0.16};
\node at (3,4) {0.39};
\node at (4,4) {0.91};
\node at (0,5) {0.15};
\node at (1,5) {0.52};
\node at (2,5) {0.37};
\node at (3,5) {0.22};
\node at (4,5) {0.12};
\node at (2,-1) {MIRT};
\node at (5,2.5) {$\cup$};
\begin{scope}[xshift=6cm]
\draw (-0.5,-0.5) rectangle ++(5,6);
\node at (0,0) {1};
\node at (1,0) {0};
\node at (2,0) {1};
\node at (3,0) {1};
\node at (4,0) {1};
\node at (0,1) {1};
\node at (1,1) {1};
\node at (2,1) {0};
\node at (3,1) {0};
\node at (4,1) {0};
\node at (0,2) {0};
\node at (1,2) {0};
\node at (2,2) {0};
\node at (3,2) {1};
\node at (4,2) {1};
\node at (0,3) {1};
\node at (1,3) {0};
\node at (2,3) {1};
\node at (3,3) {0};
\node at (4,3) {1};
\node at (0,4) {1};
\node at (1,4) {0};
\node at (2,4) {1};
\node at (3,4) {1};
\node at (4,4) {0};
\node at (0,5) {1};
\node at (1,5) {1};
\node at (2,5) {0};
\node at (3,5) {1};
\node at (4,5) {1};
\node at (2,-1) {Q-matrice};
\end{scope}
\node at (11,2.5) {$\Rightarrow$};
\begin{scope}[xshift=12cm]
\draw (-0.5,-0.5) rectangle ++(5,6);
\node at (0,0) {0.63};
\node at (2,0) {0.71};
\node at (3,0) {1.00};
\node at (4,0) {0.00};
\node at (0,1) {0.25};
\node at (1,1) {0.54};
\node at (3,2) {0.41};
\node at (4,2) {0.10};
\node at (0,3) {0.03};
\node at (2,3) {0.05};
\node at (4,3) {0.36};
\node at (0,4) {0.05};
\node at (2,4) {0.27};
\node at (3,4) {0.07};
\node at (0,5) {0.47};
\node at (1,5) {0.18};
\node at (3,5) {0.15};
\node at (4,5) {0.63};
\node at (2,-1) {GenMA};
\end{scope}
\end{tikzpicture}
\end{document}


\chapter{InitialD : une heuristique pour le démarrage à froid}
\input{intro-dpp}
\input{dpp}

\chapter{Conclusion et perspectives}
\input{perspectives}

%\chapter*{Résumé}
%\input{summary}

\clearpage
\listoffigures
\listoftables

\appendix

\chapter{Implémentation des modèles}
\input{code}

%\nocite{*}
\emergencystretch=1em
\printbibliography

\null\newpage
\thispagestyle{empty}
\input{style-abstractfr}

\null\newpage
\thispagestyle{empty}
\input{style-abstracten}

\end{document}

%%% Local Variables:
%%% mode: latex
%%% TeX-master: t
%%% End:
