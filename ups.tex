%%% This LaTeX template was created by Adrien Thurotte
%% It is approved by Paris-Saclay University (december 2015)
% And, of course, needs improvements : feel free to work on it and share it back ...
%% Under the licence CC0 - Others may freely build upon, enhance and reuse the works for any purposes without restriction under copyright or database law. https://creativecommons.org/about/cc0
%\usepackage[T1]{fontenc}
%\usepackage[latin1]{inputenc}
%\usepackage[french]{babel}
%\usepackage{graphicx} % For commande \includegraphics
%\usepackage[usenames,dvipsnames,svgnames,table]{xcolor} %% To color in violet the text
%\frenchspacing
\usepackage[absolute]{textpos} % to place elements in the page
%\usepackage{multicol} % To write summary in two columns-mode
%\usepackage{calc} % To calculate textwidth
%\RequirePackage{geometry}% That nicely create a one-page template
\usepackage{tikz} %% Make the square.

\label{The thesis}
\newcommand{\PhDTitleFR}{Modèles de tests adaptatifs pour le diagnostic de connaissances dans un cadre d'apprentissage à grande échelle} %% Titre en Français
\newcommand{\PhDkeywordsFR}{tests adaptatifs, diagnostic de connaissances, cours en ligne ouverts et massifs (MOOC), théorie de la réponse à l'item, q-matrice, analytique de l'apprentissage} %% 3 à 6 mots clefs
\newcommand{\PhDsumFR}{Cette thèse porte sur les tests adaptatifs dans les environnements d'apprentissage. Elle s'inscrit dans les contextes de fouille de données éducatives et d'analytique de l'apprentissage, où l'on s'intéresse à utiliser les données laissées par les apprenants dans des environnements éducatifs pour optimiser l'apprentissage au sens large.

L'évaluation par ordinateur permet de stocker les réponses des apprenants facilement, afin de les analyser et d'améliorer les évaluations futures. Dans cette thèse, nous nous intéressons à un certain type de test par ordinateur, les tests adaptatifs. Ceux-ci permettent de poser une question à un apprenant, de traiter sa réponse à la volée, et de choisir la question suivante à lui poser en fonction de ses réponses précédentes. Ce processus réduit le nombre de questions à poser à un apprenant tout en conservant une mesure précise de son niveau. Les tests adaptatifs sont aujourd'hui implémentés pour des tests standardisés tels que le GMAT ou le GRE, administrés à des centaines de milliers d'étudiants. Toutefois, les modèles de tests adaptatifs traditionnels se contentent de noter les apprenants, ce qui est utile pour l'institution qui évalue, mais pas pour leur apprentissage. C'est pourquoi des modèles plus formatifs ont été proposés, permettant de faire un retour plus riche à l'apprenant à l'issue du test pour qu'il puisse comprendre ses lacunes et y remédier. On parle alors de diagnostic adaptatif.

Dans cette thèse, nous avons répertorié des modèles de tests adaptatifs issus de différents pans de la littérature. Nous les avons comparés de façon qualitative et quantitative. Nous avons ainsi proposé un protocole expérimental, que nous avons implémenté pour comparer les principaux modèles de tests adaptatifs sur plusieurs jeux de données réelles. Cela nous a amenés à proposer un modèle hybride de diagnostic de connaissances adaptatif, meilleur que les modèles de tests formatifs existants sur tous les jeux de données testés. Enfin, nous avons élaboré une stratégie pour poser plusieurs questions au tout début du test afin de réaliser une meilleure première estimation des connaissances de l'apprenant. Ce système peut être appliqué à la génération automatique de feuilles d'exercices, par exemple sur un cours en ligne ouvert et massif (MOOC).} %% Résumé


\newcommand{\PhDTitleEN}{Adaptive Testing using Cognitive Diagnosis for Large-Scale Learning} %% Title
\newcommand{\PhDkeywordsEN}{adaptive testing, cognitive diagnosis, massive open online courses (MOOCs), item response theory, q-matrix, learning analytics} %% 3-6 Keywords
\newcommand{\PhDsumEN}{This thesis studies adaptive tests within learning environments. It falls within educational data mining and learning analytics, where student educational data is processed so as to optimize their learning.

Computerized assessments allow us to store and analyze student data easily, in order to provide better tests for future learners. In this thesis, we focus on computerized adaptive testing. Such adaptive tests which can ask a question to the learner, analyze their answer on the fly, and choose the next question to ask accordingly. This process reduces the number of questions to ask to a learner while keeping an accurate measurement of their level. Adaptive tests are today massively used in practice, for example in the GMAT and GRE standardized tests, that are administered to hundreds of thousands of students. Traditionally, models used for adaptive assessment have been mostly summative: they measure or rank effectively examinees, but do not provide any other feedback. Recent advances have focused on formative assessments, that provide more useful feedback for both the learner and the teacher; hence, they are more useful for improving student learning.

In this thesis, we have reviewed adaptive testing models from various research communities. We have compared them qualitatively and quantitatively. Thus, we have proposed an experimental protocol that we have implemented in order to compare the most popular adaptive testing models, on real data. This led us to provide a hybrid model for adaptive cognitive diagnosis, better than existing models for formative assessment on all tried datasets. Finally, we have developed a strategy for asking several questions at the beginning of a test in order to measure the learner more accurately. This system can be applied to the automatic generation of worksheets, for example on a massive online open course (MOOC).} %% Summary

%%%%%%%%%%%%%%%%%%%%%%%%%%%%%%%%%%%%%%%%%%%%%%%%%%%%%%%%%%%%%%%%%%%%%%%%%%%%%%%%%%%%%%%%%%%%%%%%% 
 
 \label{Personnal}
\newcommand{\PhDname}{M. Jill-Jênn Vie} % Civility, first name and name 
\newcommand{\NNT}{2016SACLC090} %% Your NNT numer (the Library will attribute one...)
\newcommand{\ecodocnum}{580} % Accrediation number
\newcommand{\ecodoctitle}{ED Sciences et technologies de l'information et de la communication} % Full name of the doctorale school
\newcommand{\PhDspeciality}{Informatique} % Speciality 
\newcommand{\PhDworkingplace}{CentraleSupélec} %PhD working place (must be one of these : Université Paris-Sud, Université de Versailles Saint Quentin, Université d’Evry, AgroParisTech, CentraleSupelec, Ecole Normale Supérieure de Cachan, Ecole Polytechnique, ENSTA ParisTech,ENSAE ParisTech, HEC,Institut d’Optique Graduate School,Telecom ParisTech or Telecom SudParis)
\newcommand{\defenseplace}{Cachan} %Place of defense
\newcommand{\defensedate}{5 décembre 2016} % Date of defense
\newcommand{\logoED}{\includegraphics[scale=1.2]{logo/STIC.png}} % Logo of doctoral school. Check the name of the correct file in the /logo folder.
\newcommand{\logoEt}{\includegraphics[scale=0.7]{logo/CentraleSupelec.png}} % Must be the logo corresponding to %PhD working place. Check the name of the correct file in the /logo folder.
\newcommand{\vpos}{1.4} %% You can modify vertical position (leave cm as unit) in order to align horizontally both images

%%%%%%In a case of "cotutelle" Delete " % " of the three following lines. Then, in style-pagedegarde.tex, delete " % " of line 3 to 5.
%\newcommand{\logoCotut}{\includegraphics[scale=1]{logo/UPS.png}} % Add here the logo of the partner. 
%\newcommand{\vCotutpos}{3.3} %% You can modify vertical position (leave cm as unit) in order to align horizontally both images
%\newcommand{\hCotutpos}{12.5} %% You can modify horizontal position (leave cm as unit) in order to align horizontally both images


%%%%%%%%%%%%%%%%%%%%%%%%%%%%%%%%%%%%%%%JURY%%%%%%%%%%%%%%%%%%%%%%%%%%%%%%%%%%%%%%%%%%%%%%%%%%%%%%%%%%%%%
% Lors du premier dépôt de la thèse le nom du président n’est pas connu, le choix du président se fait par les membres du Jury juste avant la soutenance. La précision est apportée sur la couverture lors du second dépôt.
% Choice of the jury's president is made during the defense. Thus, it must be specified only for the second file deposition in ADUM.
% All the jury members listed here must have been present during the defense.
\label{Jury}
%%% Jury member n1 (Président) %%%
\newcommand{\jurynameA}{Amel Bouzeghoub}
\newcommand{\jurygenderA}{Mme} % M. or Mme. / Mrs.
\newcommand{\juryadressA}{Télécom SudParis}
\newcommand{\jurygradeA}{Professeur}
\newcommand{\juryroleA}{Présidente du jury} % 
%%% Jury member n2 (Rapporteur) %%%
\newcommand{\jurynameB}{Nathalie Guin}
\newcommand{\jurygenderB}{Mme} % M. or Mme. / Mrs.
\newcommand{\juryadressB}{LIRIS}
\newcommand{\jurygradeB}{Maître de conférences HDR}
\newcommand{\juryroleB}{Rapporteur}
%%% Jury member n3 (Rapporteur) %%%
\newcommand{\jurynameC}{Sébastien George}
\newcommand{\jurygenderC}{M.} % M. or Mme. / Mrs.
\newcommand{\juryadressC}{Université du Maine}
\newcommand{\jurygradeC}{Professeur des universités}
\newcommand{\juryroleC}{Rapporteur}
%%% Jury member n4 (Examinateur) %%%
\newcommand{\jurynameD}{Vanda Luengo}
\newcommand{\jurygenderD}{Mme} % M. or Mme. / Mrs.
\newcommand{\juryadressD}{UPMC}
\newcommand{\jurygradeD}{Professeur des universités}
\newcommand{\juryroleD}{Examinatrice}
%%% Jury member n5 (Examinateur) %%%
\newcommand{\jurynameE}{Monique Grandbastien}
\newcommand{\jurygenderE}{Mme} % M. or Mme. / Mrs.
\newcommand{\juryadressE}{LORIA}
\newcommand{\jurygradeE}{Professeur émérite}
\newcommand{\juryroleE}{Examinatrice}
%%% Jury member n6 (Examinateur) %%%
\newcommand{\jurynameF}{Yolaine Bourda}
\newcommand{\jurygenderF}{Mme} % M. or Mme. / Mrs.
\newcommand{\juryadressF}{CentraleSupélec}
\newcommand{\jurygradeF}{Professeur}
\newcommand{\juryroleF}{Directrice de thèse}
%%% Jury member n7 (Examinateur) %%%
\newcommand{\jurynameG}{Éric Bruillard}
\newcommand{\jurygenderG}{M.} % M. or Mme. / Mrs.
\newcommand{\juryadressG}{ENS Paris-Saclay}
\newcommand{\jurygradeG}{Professeur des universités}
\newcommand{\juryroleG}{Codirecteur de thèse}
%%% Jury member n8 (Examinateur) %%%
\newcommand{\jurynameH}{Fabrice Popineau}
\newcommand{\jurygenderH}{M.} % M. or Mme. / Mrs.
\newcommand{\juryadressH}{CentraleSupélec}
\newcommand{\jurygradeH}{Professeur}
\newcommand{\juryroleH}{Coencadrant de thèse}
%% Il est possible d'ajouter des membres supplémentaires selon le même modèle.
%% More jury members can be added according to the same model
%%%%%%%%%%%%%%%%%%%%%%%%%%%%%%%%%%%%%%%COMPILATION%%%%%%%%%%%%%%%%%%%%%%%%%%%%%%%%%%%%%%%%%%%%%%%%%%%%%%%%%%%%%
\label{Document}
%% 
%% Les fichiers sont à compiler un par un. Il faut retirer la marque de commentaire " % " en début de ligne, compiler le document, puis mettre la ligne à nouveau en commentaire en vue de compiler le document suivant.
%% To compile desired document, delete the " % " at the begin of the line, compile the file, and then comment again in order to be able to compile the next one.
%%
